\documentclass[12pt]{extarticle}

\usepackage[T2A]{fontenc}
\usepackage[utf8]{inputenc}
\usepackage[ukrainian]{babel}
\usepackage{amssymb}
\usepackage{amsmath}
\usepackage[a4paper,text={19cm,27cm},centering]{geometry}
\usepackage{exercise}
\renewcommand{\ExerciseName}{Задача}

\begin{document}

\begin{Exercise}
За допомогою розкладу Адомяна розв’язати слабкосингулярне рівняння Вольтерра \[\varphi(t) = t - \dfrac{9}{4} t^{\frac{4}{3}} + \int\limits_{0}^{t} \dfrac{\varphi(s)}{(t - s)^\frac{2}{3}} \mathrm{d}s.\]
\end{Exercise}

\begin{Exercise}
За допомогою розкладу Адомяна розв’язати слабкосингулярне рівняння Вольтерра \[\varphi(t) = t - \dfrac{9}{10} t^{\frac{5}{3}} + \int\limits_{0}^{t} \dfrac{\varphi(s)}{(t - s)^\frac{1}{3}} \mathrm{d}s.\]
\end{Exercise}

\begin{Exercise}
За допомогою розкладу Адомяна розв’язати слабкосингулярне рівняння Вольтерра \[\varphi(t) = \sin t + \dfrac{3}{2}\left(\cos t - 1\right)^{\frac{2}{3}} + \int\limits_{0}^{t} \dfrac{\varphi(s)}{(\cos t - \cos s)^\frac{1}{3}} \mathrm{d}s.\]
\end{Exercise}

\begin{Exercise}
За допомогою розкладу Адомяна розв’язати слабкосингулярне рівняння Вольтерра \[\varphi(t) = \cos t - \dfrac{2}{3}\left(\sin t\right)^{\frac{2}{3}} + \int\limits_{0}^{t} \dfrac{\varphi(s)}{(\sin t - \sin s)^\frac{1}{3}} \mathrm{d}s.\]
\end{Exercise}

\begin{Exercise}
За допомогою розкладу Адомяна розв’язати слабкосингулярне рівняння Вольтерра \[\varphi(t) = e^{-t} + \dfrac{3}{2}\left(e^{-t} - 1\right)^{\frac{2}{3}} + \int\limits_{0}^{t} \dfrac{\varphi(s)}{(e^{-t} - e^{-s})^\frac{1}{3}} \mathrm{d}s.\]
\end{Exercise}

\begin{Exercise}
За допомогою розкладу Адомяна розв’язати слабкосингулярне рівняння Вольтерра \[\varphi(t) = t - \dfrac{3}{4} t^{\frac{4}{3}} + \int\limits_{0}^{t} \dfrac{\varphi(s)}{(t^2 - s^2)^\frac{1}{3}} \mathrm{d}s.\]
\end{Exercise}

\begin{Exercise}
За допомогою розкладу Адомяна розв’язати слабкосингулярне рівняння Вольтерра \[\varphi(t) = t^3 - \dfrac{8}{21}t^{\frac{7}{2}} + \int\limits_{0}^{t} \dfrac{\varphi(s)}{(t^2 - s^2)^\frac{1}{4}} \mathrm{d}s.\]
\end{Exercise}

\begin{Exercise}
За допомогою розкладу Адомяна розв’язати слабкосингулярне рівняння Вольтерра \[\varphi(t) = \dfrac{2}{3} t^3 + \int\limits_{0}^{t} \dfrac{\varphi(s)}{(t^4 - s^4)^\frac{1}{4}} \mathrm{d}s.\]
\end{Exercise}

\begin{Exercise}
За допомогою розкладу Адомяна розв’язати слабкосингулярне рівняння Вольтерра \[\varphi(t) = t^5 - \dfrac{16}{63}t^{\frac{21}{4}} + \int\limits_{0}^{t} \dfrac{\varphi(s)}{(t^3 - s^3)^\frac{1}{4}} \mathrm{d}s.\]
\end{Exercise}

\begin{Exercise}
За допомогою розкладу Адомяна розв’язати слабкосингулярне рівняння Вольтерра \[\varphi(t) = \dfrac{7}{10}t^5 + \int\limits_{0}^{t} \dfrac{\varphi(s)}{(t^3 - s^3)^\frac{1}{3}} \mathrm{d}s.\]
\end{Exercise}

\begin{Exercise}
За допомогою розкладу Адомяна розв’язати слабкосингулярне рівняння Вольтерра \[\varphi(t) = e^t - \dfrac{7}{6}\left(e^t - 1\right)^\frac{6}{7} + \int\limits_{0}^{t} \dfrac{\varphi(s)}{\left(e^t - e^s\right)^\frac{1}{7}} \mathrm{d}s.\]
\end{Exercise}

\begin{Exercise}
За допомогою розкладу Адомяна розв’язати слабкосингулярне рівняння Вольтерра \[\varphi(t) = e^t (e^t+1) - \dfrac{3}{10}\left(e^t - 1\right)^\frac{2}{3}(3e^t+7) + \int\limits_{0}^{t} \dfrac{\varphi(s)}{\left(e^t - e^s\right)^\frac{1}{3}} \mathrm{d}s.\]
\end{Exercise}

\begin{Exercise}
За допомогою розкладу Адомяна розв’язати слабкосингулярне рівняння Вольтерра \[\varphi(t) = \sin 2t + \dfrac{3}{5}\left(3 \cos t + 2\right)(\cos t - 1)^\frac{2}{3} + \int\limits_{0}^{t} \dfrac{\varphi(s)}{\left(\cos t - \cos s\right)^\frac{1}{3}} \mathrm{d}s.\]
\end{Exercise}

\begin{Exercise}
За допомогою розкладу Адомяна розв’язати слабкосингулярне рівняння Вольтерра \[\varphi(t) = 1 - \dfrac{\pi}{2} + \int\limits_{0}^{t} \dfrac{\varphi(s)}{\left(t^2 - s^2\right)^\frac{1}{2}} \mathrm{d}s.\]
\end{Exercise}

\begin{Exercise}
За допомогою розкладу Адомяна розв’язати слабкосингулярне рівняння Вольтерра \[\varphi(t) = \pi - \dfrac{\pi^2}{2} + \left(1-\dfrac{\pi}{4}\right) t^2 + \int\limits_{0}^{t} \dfrac{\varphi(s)}{\left(t^2 - s^2\right)^\frac{1}{2}} \mathrm{d}s.\]
\end{Exercise}

\end{document}